% xcolor=table is because it seems that beamer uses the xcolor package in a
% strange way and thus don't accept us giving arguments to the package.
\documentclass[slidestop,compress,mathserif, xcolor=table]{beamer}

\usepackage[T1]{fontenc}
\usepackage[utf8]{inputenc}
\usepackage[english]{babel}
\usepackage{listings}
\usepackage{hyperref}

\lstset{
    language=c,                        % choose the language of the code
    basicstyle=\ttfamily\footnotesize, % the size of the fonts that are used for the
    keywordstyle=,
    numbers=left,                      % where to put the line-numbers
    numberstyle=\footnotesize,         % the size of the fonts that are used for the
    numbersep=5pt,                     % how far the line-numbers are from the code
    backgroundcolor=\color{white},     % choose the background color. You must add
    showspaces=false,                  % show spaces adding particular underscores
    showstringspaces=false,            % underline spaces within strings
    showtabs=false,                    % show tabs within strings adding particular
    frame=single,                      % adds a frame around the code
    breaklines=true,                   % sets automatic line breaking
    breakatwhitespace=false            % sets if automatic breaks should only happen at
}

% \usepackage{mdwtab}
% \usepackage{mathenv}
% \usepackage{amsfonts}
% \usepackage{amsmath}
% \usepackage{amssymb}
% \usepackage{amsthm}

\usepackage{semantic}
\renewcommand{\ttdefault}{txtt} % Bedre typewriter font

% Use the NAT theme in uk (also possible in DK)
\usetheme[nat,uk, footstyle=low, TPrawlrimage=pony.jpg]{Frederiksberg}

\definecolor{mypink}{RGB}{255,200,220}
\setbeamercolor{background canvas}{bg=mypink}
%\setbeamercolor{structure}{fg=blue}

% Make overlay sweet nice by having different transparancy depending on how
% "far" ahead the overlay is. AWSOME!!
%\setbeamercovered{highly dynamic}
% possible to shift back, so they are just invisible untill they should overlay
% \setbeamercovered{invisible}

% Extend figures into either left or right margin
% Ex: \begin{narrow}{-1in}{0in} .. \end{narrow} will place 1in into left margin
\newenvironment{narrow}[2]{%
  \begin{list}{}{%
  \setlength{\topsep}{0pt}%
  \setlength{\leftmargin}{#1}%
  \setlength{\rightmargin}{#2}%
  \setlength{\listparindent}{\parindent}%
  \setlength{\itemindent}{\parindent}%
  \setlength{\parsep}{\parskip}}%
\item[]}{\end{list}}



\usepackage{tikz}
\usetikzlibrary{calc,shapes,arrows}
% below is for use of backgrounds (foregrounds are not used so they are not
% added to this specification).
\pgfdeclarelayer{background}
\pgfsetlayers{background,main}


\usepackage{subfigure}


% Write a short text to have that shown in the footer of slides other than the
% title slide.
\title[]{Advanced shellcode + return-oriented programming}
% A possible subslide.
%\subtitle{Regular-expression based bit coding}


\author[br0ns \and Idolf Hatler]
       {br0ns \and Idolf Hatler}

% Only write DIKU in the footer of slides (except the title slide).
\institute[DIKU]{Department of Computer Science}

% Remove the date stamp from the footer of slides (except title slide) by giving
% it no short "text"
\date[]{\today}

\begin{document}

\frame[plain]{\titlepage}

\begin{frame}[c]
    \frametitle{Aftenens Program}

    \begin{itemize}
        \item Almene værktøjer
        \item Hvor er jeg?
        \item Mon min kode er her?
        \item Hvad med over netværket?
        \item Hvor \emph{er} min kode?
        \item Hvad hvis man ikke kan få shellcode ind overhovedet?
    \end{itemize}
\end{frame}

\begin{frame}[c]
    \frametitle{Koder, kend din assembler}
    \url{http://www.nasm.us/docs.php} \\

    \begin{itemize}
        \pause\item \texttt{\%define STD\_IN 0 ; std\_in file handle}
        \pause\item \texttt{\%include "my-defines.asm"}
        \pause\item \texttt{nasm nizzle.asm -I defines/linux/}
        \pause
          \begin{itemize}
          \item GOTCHA: afsluttende skråstreg er \textit{vigtig!}
          \end{itemize}
        \pause\item \texttt{nasm nizzle.asm -D DEBUG}
        \pause\lstinputlisting{ifdef.asm}
    \end{itemize}
\end{frame}

\begin{frame}[c]
    \frametitle{Netcat: The TCP/IP Swiss army knife}
    \begin{itemize}
      \item \url{nc110.sourceforge.net}
      \item \url{www.debian-administration.org/articles/58}
      \item \texttt{man nc}
    \end{itemize}

    \pause \textbf{OBS:} Der er to udgaver af netcat. netcat-tradition og
    netcat-openbsd. De har forskellige parametre. Eksemplerne bruger
    traditional.
\end{frame}

\begin{frame}[c]
    \frametitle{Mikrolegetime: Netcat}

    \lstset{language=bash, numbers=none}
    \pause
    Forbind til vært på port 23:
    \lstinputlisting{nc-connect.sh}
    \pause
    Lyt på port 1337:
    \lstinputlisting{nc-listen.sh}
    \pause
    Send data:
    \lstinputlisting{nc-send.sh}
\end{frame}

\begin{frame}[c]
    \frametitle{Hvor er jeg?}

    \pause\lstinputlisting{move-eip.asm}

    \pause Det kan man ikke! \vskip8pt

    \pause Men man kan bruge en trampolin!

    \pause\lstinputlisting{trampolin.asm}

\end{frame}

\begin{frame}[c]
    \frametitle{Legetime 1 - skriv din egen trampolin som udskriver ``Hello
    World!''}

  \pause

    {\scriptsize
      Prøv på at bruge defines (+ includes?). \vskip8pt

      C: \texttt{write(int fd, const void *buf, size\_t counter);
        exit(0)} \vskip8pt

      Assembler: \texttt{eax = 4, ebx = 1 (stdout), ecx = buf, edx =
        count, int 0x80, eax = 1, ebx = 0, int 0x80} \vskip8pt

      \begin{tabular}{|c|p{6cm}|}
        \hline
        \texttt{[BITS 32]} & Skal stå på først linje for få nasm til at kompilere 32-bit \\\hline
        \texttt{mov $a$, $b$} & Flytter $b$ til $a$, hvor $a$ og $b$ er registre,
        hukommelse eller heltal. \\\hline
        \texttt{push r} & Pusher register \texttt{r} \\\hline
        \texttt{pop r} & Pop register \texttt{r} \\\hline
        \texttt{label:} & Laver en label man kan hoppe til (og ikke bruge til meget
        andet her) \\\hline
        \texttt{jmp label}/\texttt{call label} & Hopper/caller en label \\\hline
        \texttt{str1: db "pony",0} & Gemmer den nul-terminerede streng ``pony'' i hukommelsen med \texttt{str1} som peger til den \\\hline
      \end{tabular} \vskip8pt

      Kør med: \\\texttt{nasm -f bin filnavn.asm; ./test\_code filnavn}
    }
\end{frame}

\begin{frame}[c]
    \frametitle{Hvor mon min kode er?}

    \begin{itemize}
    \item NOP-sleds.
    \item Heap spraying.
    \end{itemize}
\end{frame}

\begin{frame}[c]
    \frametitle{Hvad med over netværk?}

    \begin{itemize}
    \item Sockets.
    \item Connect-back.
    \item Bindshell.
    \end{itemize}

    \pause
    \vskip10pt

    \textbf{Mikrolegetime:} Leg lidt med bindshell og connect-back.
    \texttt{connect-back.asm, bindshell.asm, bindshell.c}

\end{frame}

\begin{frame}[c]
    \frametitle{Legetime 2 - hack den her server}

    På 192.168.43.35 kører \texttt{server} (\texttt{server.c}) på port
    30001. Hack den!

\end{frame}

\begin{frame}[c]
    \frametitle{Hvor \emph{er} min kode?}

    \begin{itemize}
    \item Egg hunter.
    \item Omelet hunter.
    \end{itemize}

    \lstinputlisting{egg-hunter.asm}

    \pause
    \textbf{Mikrolegetime:} Brug \texttt{egg-hunter} til at køre
    \texttt{huge-shellcode}.
\end{frame}

\begin{frame}[c]
  \frametitle{Hvad hvis man ikke kan få shellcode ind overhovedet?}

  \textit{[Sketch]}
  \vskip20pt

  Kode uden kode (return-oriented programming).


  Stack-billede. \texttt{pop-ret}.

\end{frame}

\begin{frame}[c]
    \frametitle{Legetime 3}

    Løs legetime 5 fra stack overflow -oplægget.

\end{frame}

\end{document}
