% xcolor=table is because it seems that beamer uses the xcolor package in a
% strange way and thus don't accept us giving arguments to the package.
\documentclass[slidestop,compress,mathserif, xcolor=table]{beamer}

\usepackage[T1]{fontenc}
\usepackage[utf8]{inputenc}
\usepackage[english]{babel}

\usepackage{multicol}

% \usepackage{mdwtab}
% \usepackage{mathenv}
% \usepackage{amsfonts}
% \usepackage{amsmath}
% \usepackage{amssymb}
% \usepackage{amsthm}

\usepackage{semantic}
\renewcommand{\ttdefault}{txtt} % Bedre typewriter font

% Use the NAT theme in uk (also possible in DK)
\usetheme[nat,uk, footstyle=low]{Frederiksberg}

% Make overlay sweet nice by having different transparancy depending on how
% "far" ahead the overlay is. AWSOME!!
% \setbeamercovered{highly dynamic}
% possible to shift back, so they are just invisible untill they should overlay
% \setbeamercovered{invisible}

% Extend figures into either left or right margin
% Ex: \begin{narrow}{-1in}{0in} .. \end{narrow} will place 1in into left margin
\newenvironment{narrow}[2]{%
  \begin{list}{}{%
  \setlength{\topsep}{0pt}%
  \setlength{\leftmargin}{#1}%
  \setlength{\rightmargin}{#2}%
  \setlength{\listparindent}{\parindent}%
  \setlength{\itemindent}{\parindent}%
  \setlength{\parsep}{\parskip}}%
\item[]}{\end{list}}



\usepackage{tikz}
\usetikzlibrary{calc,shapes,arrows}
% below is for use of backgrounds (foregrounds are not used so they are not
% added to this specification).
\pgfdeclarelayer{background}
\pgfsetlayers{background,main}


\usepackage{subfigure}


% Write a short text to have that shown in the footer of slides other than the
% title slide.
\title[]{Zomg pwnies!}
% A possible subslide.
% \subtitle{Regular-expression based bit coding}


\author[Morten Brøns-Pedersen]
       {\includegraphics[scale=0.3]{zomg-pwnies}\\Morten Brøns-Pedersen}

% Only write DIKU in the footer of slides (except the title slide).
\institute[DIKU]{Department of Computer Science}

% Remove the date stamp from the footer of slides (except title slide) by giving
% it no short "text"
\date[]{July 18, 2010}

\begin{document}

\frame[plain]{\titlepage}

\begin{frame}
  \frametitle{Overview}

  Website: slamkode.dk\\[1em]

  This is the stuff we'll go over today:
  \begin{itemize}
  \item Introduction.
  \item Reversing.
  \item Shellcode.
  \item Stack buffer overflow.
  \item Heap buffer ovorflow and Doug Lea's malloc-and-friends.
  \item Writing exploits.
  \end{itemize}
\end{frame}

\begin{frame}
  \frametitle{Introduction}

  \begin{itemize}
  \item \alert<1>{What is our goal?}
  \item \alert<2>{What is Pwnies' teaching form?}
  \item \alert<3>{Which prerequisites should you have?}
  \item \alert<4>{What do we expect of you?}
  \item \alert<5>{What will you learn?}
  \end{itemize}

  \only<1>{Send a team from DIKU to DEFCON CTF in 2011.}

  \only<2>{We meet every Sunday as long as anyone is still motivated. A lecture
    in the morning and workshop in the afternoon. The first four lectures are
    already planned - after that \emph{you} will do the talking.}

  \only<3>{None. We'll start with the basic stuff and work our way from
    there. But you should have looked a chapters 0x100 and 0x200 in ``Hacking:
    The Art of Exploitation'' and installed IDA Pro for today.

    A lot of the ground we cover in Pwnies is similar to the curriculum of
    Proactive Computer Security (block 4 in 2011).}

  \only<4>{You should really try to be here every Sunday. Otherwise you will
    have a hard time. In addition to the lectures and workshops there will be
    some homework, both reading and exercises.}

  \only<5>{Our main focus is binary exploitation. That means a \emph{lot} of
    reversing. We really need people who can run HandRays as fast as a modern
    computer can run HexRays.

    But we cover other topics too. The first four weeks focus on
    \begin{enumerate}
    \item[Week 0] Binary exploitation (Morten).
    \item[Week 1] Forensics (Irfan).
    \item[Week 2] Crypto (Johan).
    \item[Week 3] Advanced reversing with IDA Pro (Jesper).
    \end{enumerate}
  }
\end{frame}

\begin{frame}
  \frametitle{Binary exploitation}

  For most of the problems we'll look at today we're after code execution. In
  general that involves three steps:
  \begin{itemize}
  \item \alert<1>{Get your (shell)code into memory.}
  \item \alert<2>{Get control of execution.}
  \item \alert<3>{Execute your (shell)code.}
  \end{itemize}

  \only<1>{Often this is not a big problem. For today the only requirement is
    that your shellcode does not contain any NUL bytes.}

  \only<2>{This is where you need to find a bug in the target program. A typical
    example is to overflow a stack allocated buffer and overwrite the return
    address. We'll look into this later.}

  \only<3>{Most of the time you don't have a clue where in memory your shellcode
    is located. So it can be tricky to get it executed even though you have
    control over the instruction pointer.}
\end{frame}

\begin{frame}
  \frametitle{Reversing (finding the bug)}

  It \emph{is} possible to exploit a program without a clue of how it works
  (someone\texttrademark\ should hold a lecture on fuzzing) but we will try to
  be a bit more ninja about it.\\[1em]

  At DEFCON CTF you don't get the source code for the executables. So the only
  way to know how the program works is to reverse it.\\[1em]

  Here at Pwnies we will use IDA Pro. IDA Pro is a Windows program, but it runs
  OK under wine. And it ships with a Linux debugger.
\end{frame}

\begin{frame}
  \frametitle{Focus}

  Today we will focus on
  \begin{itemize}
  \item Intel X86 32 bits architecture
  \item ELF executables.
  \item X86 assembler (Intel syntax).
  \item Linux.
  \end{itemize}

  This is a good starting point mainly because of the good documentation.
\end{frame}

\begin{frame}
  \frametitle{(Almost) general purpose registers}

  See ``The Art of Picking Intel Registers'' (\cite{swanson2003}).

  \begin{enumerate}
  \item[EAX] Accumulator. Results.
  \item[EBX] Base register (data pointer). General purpose.
  \item[ECX] Loop counter.
  \item[EDX] Data register, I/O pointer. 64-bit extension of EAX.
  \item[ESI] Source register.
  \item[EDI] Destination register.
  \item[EBP] Base pointer.
  \item[ESP] Stack pointer.
  \end{enumerate}
\end{frame}

\begin{frame}
  \frametitle{Register addressing}

  Example: EAX is 32 bits, AX is the lower 16 bits of EAX, AH is the upper 8
  bits of AX and AL is the lower 8 bits of AX.
\end{frame}

\begin{frame}
  \frametitle{Other registers}

  \begin{enumerate}
  \item[EIP] Instruction pointer. Can't be controlled directly.
  \item[Eflags] Indicate events. For example ZF (zero flag), SF (sign
    flag) and CF (carry flag). See ``Intel Architecture Software Developer's
    Manual, Volume 1'' (\cite{intel1999-vol1}).
  \end{enumerate}
\end{frame}

\begin{frame}
  \frametitle{Segment registers}

  There are six: CS, DS, SS, ES, FS and GS.\\[1em]

  Addresses are 48 bits: a 16 bit segment selector and a 32 bit offset. In
  reality segmentation is practically disabled. Code is data and data is code,
  pointers are 32 bits.\\[1em]

  Segment FS is an exception, it might change from thread to thread.
\end{frame}

\begin{frame}
  \frametitle{Types}

  \begin{itemize}
  \item Byte: 8 bits.
  \item Word: 2 bytes, 16 bits.
  \item Doubleword (or dword): 2 words, 32 bits.
  \item Quadword (or qword): 2 dwords, 64 bits.
  \item Doublequadword: 2 qwords, 128 bits.
  \end{itemize}
\end{frame}

\begin{frame}
  \frametitle{Tools}

  \begin{description}
  \item[IDA Pro] The tool, hands down. We will use IDA Pro from day 0, so if you
    don't have it installed already: Get it! Week 3 will focus on using IDA Pro.
  \item[objdump and gdb] OK tools if you just need something lightweight. Good
    for working over ssh, for example.
  \end{description}
\end{frame}

\begin{frame}
  \frametitle{Intel X86 assembler}

  \begin{itemize}
  \item Get a copy of ``Intel Architecture Software Developer's Manual, Volume
    2: Instruction Set Reference'' (\cite{intel1999-vol2}).
  \item gdb uses AT\&T syntax, we use Intel syntax.
  \item Little endian (least significant byte has lowest address).
  \end{itemize}
\end{frame}

\begin{frame}
  \frametitle{Some OP codes}

  \only<+>{
    \begin{description}
    \item[PUSH src] Pushes it's operand onto the stack. Decreases ESP according to the
      operand. If ESP itself is pushed, it is the value of ESP \emph{before} the
      operation that gets pushed.
    \item[POP dst] Pops a value from the stack and stores it in the operand. Increases
      ESP accordingly.
    \item[MOV dst, src] Moves the value of src to dst. The source operand can be
      a value, a register or a memory location (using the [reg1 + reg2 * imm1 +
      imm2] notation). The destination can be a register or a memory location.
    \item[ADD dst, src] Adds src and dst and stores the result in dst.
    \item[SUB dst, src] You guessed it.
    \end{description}
  }

  \only<+>{
    \begin{description}
    \item[CMP op1, op2] Compares op1 to op2 by subtracting op2 from op1, and sets
      CF, OF, SF, ZF, AF and PF in Eflags accordingly. Normally used in
      conjunction with jump instructions.
    \item[JZ, JNZ, JL, JLE, JG, JGE] Jump according to Eflags. If Eflags was set
      by CMP op1 op2, then the OP codes mean jump if op1 is equal to, not equal
      to, less than, less than or equal to, greater than or greater than or equal
      to op2 respectively. There are many other OP codes in the Jcc family.
    \item[JMP dst] Jump to dst (register, memory, offset or absolute address).
    \item[CALL dst] Push the address of the following OP code on the stack and
      jump to dst.
    \item[RET] Pop the return address and jump to it.
    \end{description}
  }

  \only<+>{
    Look these up yourself:
    \begin{list}{}
    \item LEA, PUSHA, POPA, PUSHF, POPF, CDQ, AND, OR, NOT, XOR, BSF, BSR,
      BSWAP, XCHG, MUL, DIV, NEG, NOP, INC, DEC, ENTER, LEAVE, LOOP, STOS, ROL,
      ROR, SHL, SHR, XADD, XLAT, and many more...
    \end{list}
  }
  \end{frame}

\begin{frame}
  \frametitle{ELF and virtual memory}

  ELF is a format for storing executables, object files and shared
  libraries.\\[1em]

  The ELF file describes how the memory should be laid out (what goes where),
  and how linking is to be done.\\[1em]

  Our programs run in virtual memory meaning that every program ``thinks'' that
  is has its own (rather large) chunk of memory. Direct memory references is possible.
\end{frame}

\begin{frame}
  \frametitle{The stack}

  The stack grows down from higher addresses to lower addresses.\\[1em]

  The top of the stack is pointed to by ESP. To allocate space on the stack
  \emph{subtract} from ESP, and to free space on the stack \emph{add} to
  it.\\[1em]

  Remember ESP points to the last value pushed.
\end{frame}

\begin{frame}
  \frametitle{Function calls}

  Refer to \cite{erickson2008} for details.\\[1em]

  The way a function call is done is the caller first pushes the arguments
  (right to left) and the return address onto the stack. Then the callee pushes
  EBP onto the stack and stores ESP in EBP. Local variables are allocated on the
  stack by subtracting from ESP.

  After the function has executed ESP and EBP is brought back to normal.\\[1em]

  Try and write a very small program and reverse it to see how it all works!
\end{frame}

\begin{frame}
  \frametitle{The heap}

  The heap is used for dynamically allocated memory. The heap is managed by a
  memory allocator. There are many implementations. Later today we will learn
  how to exploit Doug Lee's allocator (usually just called dlmalloc).
\end{frame}

\begin{frame}
  \frametitle{System calls}

  The program communicate with the operating system by system calls. System
  calls on Linux are particularly simple.\\[1em]

  EAX holds the system call number, and EBX, ECX, EDX, ESI, EDI and EBP hold the
  arguments. When the registers are set up right interrupt 0x80 executes the
  system call. The OP code is INT 0x80.\\[1em]

  See manpage syscalls(2).
\end{frame}

\begin{frame}[fragile]
  \frametitle{Examples}

\tiny
\begin{figure}
\begin{multicols}{2}
If-statement
\begin{verbatim}
080483e4 <main>:
 80483e4: push   ebp
 80483e5: mov    ebp,esp
 80483e7: and    esp,0xfffffff0
 80483ea: sub    esp,0x10
 80483ed: cmp    DWORD [ebp+0x8],0x1
 80483f1: jg     80483ff <main+0x1b>
 80483f3: mov    DWORD [esp], 0x80484d0
 80483fa: call   8048318 <puts@plt>
 80483ff: mov    eax,0x0
 8048404: leave
 8048405: ret








\end{verbatim}
Loop
\begin{verbatim}
080483b4 <main>:
 80483b4:	push   ebp
 80483b5:	mov    ebp,esp
 80483b7:	sub    esp,0x70
 80483ba:	mov    DWORD [ebp-0x4],0x0
 80483c1:	jmp    80483df <main+0x2b>
 80483c3:	mov    eax,[ebp-0x4]
 80483c6:	mov    edx,[ebp+0xc]
 80483c9:	add    edx,0x4
 80483cc:	mov    ecx,[edx]
 80483ce:	mov    edx,[ebp-0x4]
 80483d1:	lea    edx,[ecx+edx]
 80483d4:	movzx  BYTE edx,[edx]
 80483d7:	mov    [ebp+eax-0x68],dl
 80483db:	add    DWORD [ebp-0x4],0x1
 80483df:	cmp    DWORD [ebp-0x4],0x63
 80483e3:	jle    80483c3 <main+0xf>
 80483e5:	mov    eax,0x0
 80483ea:	leave
 80483eb:	ret
\end{verbatim}
\end{multicols}
\end{figure}
\end{frame}

\begin{frame}[fragile]
  \frametitle{Relative jumps}

  What does these OP codes do?

\begin{verbatim}
deadbeef: eb 00    jmp 0x0
deadbef1: eb fe    jmp -0x2
\end{verbatim}

  What about this one?
\begin{verbatim}
decafbad: e8 ff ff ff ff   call -0x1
decafbb3: c0 58            [jiberish]
\end{verbatim}
\end{frame}

\begin{frame}[fragile]
  \frametitle{Gcc stack protector}

  (AT\&T syntax for the heck of it)
\tiny
\begin{verbatim}
08048404 <main>:
 8048404:	55                   	push   %ebp
 8048405:	89 e5                	mov    %esp,%ebp
 8048407:	83 e4 f0             	and    $0xfffffff0,%esp
 804840a:	83 c4 80             	add    $0xffffff80,%esp
 804840d:	8b 45 0c             	mov    0xc(%ebp),%eax
 8048410:	89 44 24 0c          	mov    %eax,0xc(%esp)
 8048414:	65 a1 14 00 00 00    	mov    %gs:0x14,%eax
 804841a:	89 44 24 7c          	mov    %eax,0x7c(%esp)
 804841e:	31 c0                	xor    %eax,%eax
 8048420:	c7 44 24 14 00 00 00 	movl   $0x0,0x14(%esp)
 ...
 8048456:	8b 54 24 7c          	mov    0x7c(%esp),%edx
 804845a:	65 33 15 14 00 00 00 	xor    %gs:0x14,%edx
 8048461:	74 05                	je     8048468 <main+0x64>
 8048463:	e8 d8 fe ff ff       	call   8048340 <__stack_chk_fail@plt>
 8048468:	c9                   	leave
 8048469:	c3                   	ret
\end{verbatim}
\end{frame}

\begin{frame}
  \frametitle{Nasty stuff}

  \begin{itemize}
  \item It is possible to detect the presence of ptrace (used by gdb). See ``A
    Little Journey to the Wonderful World of ptrace.'' (\cite{ptrace-journey}).
  \item Jumps to the middle of instructions. Ex: MOV eax,0xd4ff1234 has OP code
    b8 34 12 ff d4. The OP code of CALL esp is ff d4 which we find three bytes
    in.
  \item There's nothing natural about the way gcc (and other compilers) builds
    an ELF file. It is possible to produce a statically linked and stripped
    executable. That means (almost) no symbols and sections, just a big chunk of
    code.
  \end{itemize}
\end{frame}

\begin{frame}
  \frametitle{Shellcode}

  A shellcode is a small piece of machine code. It is typically designed to give
  the attacker a shell, which she can use to access the host.\\[1em]

  A shellcode is \emph{not} an executable. You operating system will not be able
  to run it as-is. Use the program ``demo'' (download from the website) to test your
  shellcode.

\end{frame}

\begin{frame}
  \frametitle{A typical shellcode is}

  \begin{itemize}
  \item Tiny - often less than 100 bytes.
  \item NUL byte free (in a C string a NUL byte means the end of the string).
  \item Position independent (most of the time you don't know where your
    shellcode ends up, and when you do its program specific).
  \item Relies as little as possible on its environment (memory layout, memory
    position, etc.).
  \end{itemize}

  Sometimes a host will reject certain input. For example non-printable
  characters. It is possible to write shellcode using only printable
  characters. The Internet is filled with coders/decoders.
\end{frame}

\begin{frame}
  \frametitle{Kinds of shellcode}

  See Wikipedia (\cite{wiki-shellcode}). And for some really leet stuff see
  ``English Shellcode'' (\cite{mason2009}).

  \begin{itemize}
  \item Local (just run execve(/bin/sh)).
  \item Remote.
    \begin{itemize}
    \item Bindshell (attacker connects).
    \item Connect-back (shellcode connects).
    \end{itemize}
  \item Staged. A small shellcode locates (and/or assembles) the real
    shellcode. An egg-hunter searches memory for the shellcode. Sometimes
    the shellcode's position can be calculated from pointers in known locations.
  \item Downloader.
  \end{itemize}
\end{frame}

\begin{frame}
  \frametitle{Architecture}

  As mentioned we focus on Intel X86 32 bits processors. Officially it's called
  IA32 (\cite{wiki-ia32}), but is also referred to as X86 or i386.\\[1em]

  The AMD64 architecture is referred to as X86-64, IA32e, EM64T or X64. It is
  \emph{not} the same as IA64 (used in Intel Itanium).
\end{frame}

\begin{frame}
  \frametitle{Position independent shellcode}

  Often we don't know where the shellcode is located.\\[1em]

  We have two options:
  \begin{itemize}
  \item Find the absolute address.
  \item Not rely on knowing the address.
  \end{itemize}
\end{frame}

\begin{frame}[fragile]
  \frametitle{Finding the address}

  This would be great:
\begin{verbatim}
bedab055: mov eax,eip
...
bedab100: add eax,0x1ab
...
bedab200: [DATA]
\end{verbatim}

  But there is not such an instruction in X86.
\end{frame}

\begin{frame}[fragile]
  \frametitle{Trampoline}

\begin{verbatim}
da7aba5e: eb 5e             jmp 0x5e
...
da7aba71: 58                pop eax
...
da7ababe: e8 9d ff ff ff    call -0x63
\end{verbatim}
\end{frame}

\begin{frame}[fragile]
  \frametitle{The NOP sled}

  If we don't know the exact address of the shellcode but have an idea of where
  it is we can use a NOP sled.

\begin{verbatim}
90   NOP
90   NOP
90   NOP
90   NOP
...
90   NOP
90   NOP
     [shellcode]
\end{verbatim}
\end{frame}

\begin{frame}[fragile]
  \frametitle{Pushing a string on the stack}

  If we need a string and we don't know where the shellcode is loaded, we can
  store the string in a known location. The stack is a good option.

\begin{verbatim}
bedabb1e: 68 73 73 77 64    push "sswd"
bedabb23: 68 63 2f 70 61    push "c/pa"
bedabb28: 68 0a 2f 65 74    push (0xa | ("/et" << 8))
bedabb2d: 68 6f 72 6c 64    push "orld"
bedabb32: 68 6f 2c 20 57    push "o, W"
bedabb37: 68 48 65 6c 6c    push "Hell"
bedabb3c: 89 e0             mov eax,esp
\end{verbatim}

  Register EAX points to the string ``Hello, World\textbackslash n/etc/passwd''.
\end{frame}

\begin{frame}[fragile]
  \frametitle{Controlling registers}

  \tiny
  \begin{multicols}{2}
    Setting EAX to zero
\begin{verbatim}
5a11b0a7: b8 00 00 00 00    mov eax,0x0

600dbea7: 31 c0             xor eax,eax

7bea71e5: 29 c0             sub eax,eax

da7aba5e: b8 ff ff ff ff    mov eax,-0x1
da7aba63: 40                inc eax

badbaded: 89 d8             mov eax,ebx
badbadef: 31 d8             xor eax,ebx
\end{verbatim}
    Setting EDX to zero if EAX is non-negative
\begin{verbatim}
601dc0de: 99                cdq
\end{verbatim}
    Setting EAX, ECX and EDX to zero
\begin{verbatim}
be5770ad: 31 c9             xor ecx,ecx
be5770af: f7 e9             imul
(ECX := 0, EDX:EAX := EAX * ECX)




\end{verbatim}
    Setting EAX to an arbitrary value
\begin{verbatim}
17570a57: b8 33 22 11 00    mov eax,0x00112233

fa57c0de: b8 32 23 10 01    mov eax,0x01102332
fa57c0e3: 35 01 01 01 01    xor eax,0x01010101

c01dca75: b8 03 32 21 10    mov eax,0x01122330
c01dca7a: c1 c8 04          ror eax,0x4

deadca75: b8 cc dd ee ff    mov eax,0xffeeddcc
deadca7a: f7 d0             not eax
\end{verbatim}
    Setting a register to a value less than 128
\begin{verbatim}
1116e718: 6a 17             push byte 0x17
1116e718: 58                pop eax
\end{verbatim}
    Zero if even
\begin{verbatim}
        : c1 e0 1f          shl eax,31
\end{verbatim}
    Zero if positive
\begin{verbatim}
        : c1 e8 1f          shr eax,31

        : c1 f8 1f          sar eax,31
\end{verbatim}
  \end{multicols}
\end{frame}

\begin{frame}
  \frametitle{Stack buffer overflow exploits}

  Read ``Smashing the Stack for Fun and Profit'' (\cite{alephone1996}).\\[1em]

  Local variables are allocated on the stack. In the typical example a buffer is
  overflowed in order to overwrite the return address.
\end{frame}

\begin{frame}[fragile]
  \frametitle{Inside a function}

{\tiny
\begin{verbatim}
int myfun (int a, int b, inc) {
  int x, y;
  ...
\end{verbatim}
\begin{verbatim}
cc cc cc cc    [ebp - 0x10] : third argument
bb bb bb bb    [ebp - 0xc]  : second argument
aa aa aa aa    [ebp - 0x8]  : first argument
dd cc bb aa                 : return address (0xaabbccdd)
88 77 ff bf                 : saved base pointer (0xbfff7788)
xx xx xx xx    [ebp + 0x4]  : first local variable
yy yy yy yy    [ebp + 0x8]  : second local variable
\end{verbatim}
}
Look out for strcpy and friends, and loops controlled by input.
\end{frame}

\begin{frame}
  \frametitle{Sweet!}

  {\small
    \texttt{gdb --args ./stack `perl -e 'print "A"x1030'`}\\
    \texttt{(gdb) run}
  }\\[1em]

  (A good sign that you're controlling EIP)\\[1em]

  {\small
    \texttt{Program received signal SIGSEGV, Segmentation fault.}\\
    \texttt{0x41414141 in ?? ()}
  }
\end{frame}

\begin{frame}
  \frametitle{Now what?}

  So we control EIP, now what?\\[1em]

  If the stack is not randomised we can use a NOP sled and return
  directly.\\[1em]

  Nowadays -- unfortunately -- the stack randomised. We don't know where it's
  located.
\end{frame}

\begin{frame}[fragile]
  \frametitle{Return to library}

  If we now the address of a library we can return to it, (mis)using OP
  codes. Jump and call instructions can be useful. Especially JMP esp and CALL
  esp.\\[1em]

  If you can't find the code you need, you can sometimes build it from several
  places in the library. Example (EDX holds the address of the shellcode):

\tiny
  \begin{multicols}{2}
\begin{verbatim}
08048526:   89 d0   mov eax,edx
08048528:   5e      pop esi
08048529:   c3      ret
...
080486d7:   ff d0   call eax
\end{verbatim}
\begin{verbatim}
...
d7 86 04 08
41 41 41 41
26 85 04 08  < ESP
\end{verbatim}
  \end{multicols}
\end{frame}

\begin{frame}[fragile]
  \frametitle{Jump to EAX}

\tiny
\begin{verbatim}
ff e0               jmp eax


ff d0               call eax


50                  push eax
c3                  ret


89 c3               mov ebx,eax
...
ff d3               call ebx


60                  pusha
...
81 ec 1c 00 00 00   sub esp,0x1c
...
c3                  ret
\end{verbatim}
\end{frame}

\begin{frame}
  \frametitle{Be creative!}

  OP codes are everywhere, and some codes have multiple meanings depending on
  alignment. Sometimes you will find useful OP codes in text string or
  icons. Look around.
\end{frame}

\begin{frame}
  \frametitle{Stack cookies}

  A stack cookie (or a stack canary as it is often called) is a value pushed
  onto the stack at the beginning of a function call. After the function has
  executed the value is inspected. If it has changed the program aborts. GCC and
  Microsoft Visual Studio implements stack cookies since 1997.
\end{frame}

\begin{frame}
  \frametitle{Heap buffer overflow exploits (using dlmalloc)}

  See ``A Memory Allocator'' (\cite{lea1996}).

  You will find the source code for Doug Lea's memory allocator in malloc.c in
  todays exercises.\\[1em]

  Memory allocators must find a balance between speed and resilience to
  errors. Speedy allocators are good for us.\\[1em]

  A memory allocator must have some internal data structure to keep track of
  allocated memory. If we can tamper with this data structure we might be able
  to get the allocator to help us.\\[1em]

  I will not give a recipe for exploiting dlmalloc. Look at the source code,
  it's nicely documented.
\end{frame}

\begin{frame}[fragile]
  \frametitle{Writing exploits}

  Write self-documenting maintainable code.

\tiny
\begin{verbatim}
#include <stdint.h>
...

unsigned char shellcode[] = {0x6A, 0x0E, 0x5A, ...};
struct {
  unsigned char buf[64];
  uint32_t ebp;
  uint32_t ret0;
  uint32_t esi;
  uint32_t ret1;
  char nul;
} __attribute__((packed)) overflow;

memset(&overflow, 'A', sizeof(overflow));
memcpy(overflow.buf, shellcode, sizeof(shellcode));
overflow.ret0 = 0x08048526;
overflow.ret1 = 0x080486D7;
overflow.nul = '\0';
...
\end{verbatim}
\end{frame}

\begin{frame}
  \frametitle{Other stuff}

  See \cite{erickson2008, koziol2004}.

  \begin{itemize}
  \item Format string exploits.
  \item Returning into libc (and libc chaining).
  \end{itemize}
\end{frame}

\begin{frame}
  \frametitle{Commands}

  \begin{description}
  \item[objdump -d] Disassemble.
  \item[objdump -D] Disassemble all.
  \item[objdump -h] List section header.
  \item[nasm -f elf] Assemble ELF object file.
  \item[gcc -m32 -fno-stack-protector] Compile to X86 32 bits without stack
    protection.
  \item[grep "`perl -e 'print \textbackslash "\textbackslash xff\textbackslash
    xd4\textbackslash "'`" $\lbrack$ file $\rbrack$]
    Look for CALL esp in file.
  \item[...]
  \end{description}
\end{frame}

\begin{frame}
  \frametitle{Proactive Computer Security}

  Block 4, 2011.\\[1em]

  Look it up on Absalon. Lots of useful slides and links.
\end{frame}

\begin{frame}[allowframebreaks]
  \frametitle{Bibliography}
  \tiny
  \bibliographystyle{../../literature/bibliography/theseurl}
  \bibliography{../../literature/bibliography/bibliography}
\end{frame}

\end{document}





%%% Local Variables: 
%%% mode: latex
%%% TeX-master: t
%%% End: 
